\documentclass{article}
\usepackage{blindtext}
\usepackage[T1]{fontenc}
\usepackage[utf8]{inputenc}
\usepackage{amsmath}
\usepackage{amsfonts}
\usepackage{color}
\usepackage{graphicx}
\usepackage{vmargin}

\setmargins{2.5cm}       % margen izquierdo
{1.5cm}                        % margen superior
{16.5cm}                      % anchura del texto
{23.42cm}                    % altura del texto
{10pt}                           % altura de los encabezados
{1cm}                           % espacio entre el texto y los encabezados
{0pt}                             % altura del pie de página
{2cm}                           % espacio entre el texto y el pie de página


\makeatletter
\newcommand*{\bigcdot}{}% Check if undefined
\DeclareRobustCommand*{\bigcdot}{%
  \mathbin{\mathpalette\bigcdot@{}}%
}
\newcommand*{\bigcdot@scalefactor}{.5}
\newcommand*{\bigcdot@widthfactor}{1.15}
\newcommand*{\bigcdot@}[2]{%
  % #1: math style
  % #2: unused
  \sbox0{$#1\vcenter{}$}% math axis
  \sbox2{$#1\cdot\m@th$}%
  \hbox to \bigcdot@widthfactor\wd2{%
    \hfil
    \raise\ht0\hbox{%
      \scalebox{\bigcdot@scalefactor}{%
        \lower\ht0\hbox{$#1\bullet\m@th$}%
      }%
    }%
    \hfil
  }%
}
\makeatother

\title{Tarea 1: Vectores}
\author{Jesus E. Loera}
\date{\today}

\begin{document}

\maketitle      			% Hoja con el titulo, autor y fecha

\tableofcontents			% Índice

\begin{center}
	\rule[0mm]{150mm}{0.1mm}		% Para dibujar una linea horizontal de
									% [elevación]{longitud}{grosor}
	\end{center}
	
	
\begin{abstract}			% ABSTRACT

	\noindent 				% Anula la sangria de este parrafo
	En ésta tarea demostraremos algunas de las propiedades de los vectores en 
	$\mathbb{R}^{n}$
	\end{abstract}
	
\begin{center}
	\rule[0mm]{150mm}{0.1mm}
	\end{center}

\section*{Introduction}		% Introduce el primer tema, pero el asterisco le 
                            % quita la enumeración
                            
    Este es un resumen con los apuntes más importantes de la tercera sesión del curso de mecánica teórica

\section{Transpuesta de una Matriz}


    Dada una matriz $ A = (a_{ij})_{3x3} $

    \begin{equation}
        A= \begin{bmatrix}
                a_{11} & a_{12} & a_{13} \\
                a_{21} & a_{22} & a_{23} \\
                a_{31} & a_{32} & a_{33} 
           \end{bmatrix}
    \end{equation}

    Definimos su matriz transpuesta como:

    \begin{equation}
      A^{T}= \begin{bmatrix}
              a_{11} & a_{21} & a_{31} \\
              a_{12} & a_{22} & a_{32} \\
              a_{13} & a_{23} & a_{33} 
      \end{bmatrix}
    \end{equation}

    Es posible generar la transposición de una matriz de la siguiente manera:
    \vspace{5px}

    Dada una matriz A:

    \begin{equation}
      A = (a_{ij})_{mxn} \in M_{mxn}
    \end{equation}

    \vspace{5px}
    
    Entonces su transpuesta es: 

    \begin{equation}
      A^{T}=  (b_{ij})_{nxm} \in M_{nxm} \mid b_{ij} = a_{ji} 
    \end{equation}

\section{Matrices simétricas}

    Una matriz cuadrada A es simétrica si:

    \begin{equation*}
      A^{T}=A
    \end{equation*}

\section{Matrices antisimétricas}

    Una matriz cuadrada A es antisimétrica si:

    \begin{equation*}
      A^{T}=-A
    \end{equation*}
    
\section{Definición del producto punto con multiplicación matricial}


  Recordemos la anterior definicón que teníamos para el producto punto:

  \begin{gather*}
    \bigcdot : \mathbb{R}^{n} \times \mathbb{R}^{n} \longrightarrow \mathbb{R}^{n} \\
    (X, Y) \longmapsto X \bigcdot Y =  ( x_{1}y_{1}+x_{2}y_{2}+x_{3}y_{3}+...,x_{n}y_{n} ) 
  \end{gather*}

  Dados X, Y vectores renglones en $\mathbb{R}^{n}$ o matrices de $M_{1xn}$  

  \begin{equation}
    X \bigcdot Y = XY^{T}=  \begin{bmatrix}
                              x_{1} & \cdots & x_{n}
                            \end{bmatrix}
                            \begin{bmatrix}
                              y_{1} \\
                              \vdots \\
                              y_{n}
                            \end{bmatrix}
  \end{equation}

  Obs: Notemos que con ambas formas de el producto obtenemos un escalar y una matriz 1X1

\section{Otra forma de reescribir una matriz}

  Consideremos la matriz de la siguiente matriz:

  \begin{equation}
      A= \begin{bmatrix}
              a & b & c \\
              d & e & f \\
              g & h & i 
        \end{bmatrix}
  \end{equation}

  Realizando las respectivas operaciones podremos comprobar que:

  \begin{equation}
    A=a \thinspace {i}^{T} {i} + b \thinspace {i}^{T} {j} + c \thinspace {i}^{T} {k} +
      d \thinspace {j}^{T} {i} + e \thinspace {j}^{T} {j} + f \thinspace {j}^{T} {k} +
      g \thinspace {k}^{T} {i} + h \thinspace {k}^{T} {j} + i \thinspace {k}^{T} {k} 
  \end{equation}

  Tal que:

  \begin{gather}
    i = \begin{bmatrix}
          1 & 0 & 0
        \end{bmatrix} \\
    j = \begin{bmatrix}
          0 & 1 & 0
        \end{bmatrix} \\
    k = \begin{bmatrix}
          0 & 0 & 1
        \end{bmatrix}
  \end{gather}

\section{Determinante de una matriz}

  Tengamos una matriz $A = (a_{ij})_{2X2}$

  \begin{equation}
    A = \begin{bmatrix}
          a & b \\
          c & d
        \end{bmatrix}
  \end{equation}

  Su determinante, denotado por $det(A)$ o $ \mid A \mid $, lo calculamos de la siguiente manera:

  \begin{equation}
    \mid A \mid = det(A) = A = \begin{vmatrix}
                                  a & b \\
                                  c & d
                                \end{vmatrix} = ad-bc
  \end{equation}

  Ahora, en el caso de una matriz $A = (a_{ij})_{3X3}$, por ejemplo, la de la ecuación (6):

  \begin{equation*}
    A= \begin{bmatrix}
            a & b & c \\
            d & e & f \\
            g & h & i 
      \end{bmatrix}
  \end{equation*}

Podemos calcular su determinante de la siguiente manera:

\begin{equation}
  \mid A \mid = det(A) = A= \begin{vmatrix}
                                a & b & c \\
                                d & e & f \\
                                g & h & i 
                            \end{vmatrix} 
                            =a\begin{vmatrix}
                              e & f \\
                              h & i
                            \end{vmatrix} -
                            b\begin{vmatrix}
                              d & f \\
                              g & i
                            \end{vmatrix} +
                            c\begin{vmatrix}
                              d & e \\
                              g & h
                            \end{vmatrix} 
\end{equation}

\section{Producto cruz de vectores}

  Consideremos dos vectores $ X = (x_{1},x_{2},x_{3}) ,Y=(y_{1},y_{2},y_{3}) \in \mathbb{R}^{3}$ , definimos su producto cruz como:

  \begin{gather*}
    \times : \mathbb{R}^{3} \times \mathbb{R}^{3} \longrightarrow \mathbb{R}^{3} \\
    (X, Y) \longmapsto X \times Y =  \begin{vmatrix}
                                          i     & j     & k     \\
                                          x_{1} & x_{2} & x_{3} \\
                                          y_{1} & y_{2} & y_{3} 
                                      \end{vmatrix} 
  \end{gather*}

	
\end{document}